\documentclass[bibtotoc]{scrartcl}   %bibtotoc ist daf?r da, dass das Literaturverzeichnis auch im Inhaltsverzeichnis angezeigt wird, liststotoc ist f?r Abbildungsverzeichnis und Tabellenverzeichnis im Inhaltsverzeichnis
%\pagestyle{headings} %Kapitel?berschrift und Seitennummer in Kopfzeile (au?er dort, wo neues Kapitel beginnt



%Paket für ein deutsches Literaturverzeichnis
% \usepackage{natbib}
% \bibliographystyle{agsm}

%Paket für ein deutsches Literaturverzeichnis
\usepackage[authoryear]{natbib}

% \bibliographystyle{natdin}
\bibliographystyle{natdin}

\bibliographystyle{plainnat}

%\setcounter{secnumdepth}{4} %f?r die Auflistungstiefe im Inhaltsverzeichnis. So w?rde jetzt bis zur 4. Unterteilung alles aufgelistet.
%\setcounter{tocdepth}{4}
\usepackage[utf8x]{inputenc}
%\usepackage[applemac]{inputenc} %setzt Sonderzeichen um
\usepackage{lmodern} %f?r die Trennung der W?rter
\usepackage[ngerman]{babel} % f?r die deutsche Beschriftung von z.B. Grafiken
\usepackage[babel, german=quotes]{csquotes} %Erzeugung von Anführungszeichen durch \enquote{text}
\usepackage[T1]{fontenc}
\usepackage{longtable} % erm?glicht mehrseitige Tabellen
\usepackage{booktabs} % entzerrt die Tabellenzeilen und bietet verschiedene dicke Unterteilungslinien
\usepackage{graphicx} %um Grafiken einbinden zu k?nnen
% \usepackage{float} % unterst?tzt bei der Positionierung von Grafiken und Tabellen
\usepackage{color}
\definecolor{black}{gray}{0}% 10% gray
\usepackage[colorlinks=true, linkcolor=black, urlcolor=black, citecolor=black]{hyperref}  %verwandelt alle Kapitel?berschriften, Verweise aufs Literaturverzeichnis und andere Querverweise in PDF Hyperlinks und stellt sie angenehm unauff?llig dar
\usepackage[normalem]{ulem} % l?sst Unterstreichungen umbrechen
\usepackage{textcomp}
\usepackage{multirow}
\usepackage{pdfpages} %zum Einbinden von PDFs
\usepackage{url}
\usepackage{colortbl}
\usepackage{setspace} %um den Zeilenabstand zu regulieren
\usepackage{geometry}
\geometry{a4paper, left=3cm, right=2cm, top=2cm}
\usepackage{paralist} %braucht man für formatierte Aufzählungen. Mit \begin{compactitem} \item ... und \end{compactitem} bekommt man keine Leerzeichen zwischen den aufgezählten Punkten



\onehalfspacing

%\hypersetup{linkcolor=black, urlcolor=black} %um URLs nicht rosa anzeigen zu lassen


%\usepackage{tocloft}

%\renewcommand{\cftfigpresnum}{Abb. }
%\renewcommand{\cftfigaftersnum}{:}

%% Befehl fur Literaturverweise mit Seitenangabe
%% Bsp: \site{Rupp2004}{213}
\newcommand{\site}[2]{
  \citep[S. #2]{#1}
}

% TODO Box
\usepackage{mdframed}
\newcommand{\todo}[1]{
  \vspace{0.2cm}
  \begin{mdframed}[skipabove=3pt, skipbelow=3pt, backgroundcolor=green]
    TODO: #1
  \end{mdframed}
  \vspace{0.2cm}
}

\date{\today}


\begin{document}

%!TEX root = ../hausarbeit.tex
\begin{titlepage}



%Logo der Fachhochschule Köln
\begin{figure}[!ht]
	\centering
		\includegraphics[natwidth=920pt, natheight=95pt, width=1.0\textwidth]{Bilder/LogoUni.png}


\end{figure}

\begin{center}

\vspace{0,8cm}

%vorgelegt an der...
\begin{large}
Universität zu Köln\\
Humanwissenschaftliche Fakultät\\
Erziehungswissenschaft (1-Fach)\\
BM 1 - Grundlagen der Erziehungswissenschaft\\
\vspace{1cm}
\begin{scshape}
Hausarbeit\\ 
\end{scshape}
\end{large}


\vspace{1.5cm}

%Deutscher Titel
\begin{rmfamily}
\textbf{\huge Die Waldorfpädagogik}\\
\LARGE -\\
	Ihre Wirkung im Licht der Empirie\\
\normalsize
\end{rmfamily}

\vspace{1.7cm}
%
%%Englischer Titel
%\begin{rmfamily}
%\textbf{\LARGE classic vs. agile project management}\\
%\large -\\
%	Compare klassic and agiel procedure model relating to their suitability in the media sector, especially in 3D projects
%\normalsize
%\end{rmfamily}

%\vspace{1.2cm}

\begin{large}
Seminar: Reformpädagogik und Schule\\ [0.8em]
\end{large}

%\vspace{1.0cm}

%Autor der Bachelorarbeit und die Prüfer
\begin{tabular}{rl}
        Seminarleiter:  &  Dr. Claus Dahlmanns\\
       			       &  \small Universität zu Köln \\[0.8em]
\end{tabular}
							  
\begin{large}
Wintersemester 2012/2013\\
Abgabe der Arbeit: xx.03.2013\\
\end{large}

\vspace{1.7cm}

%\end{center}

%\newpage
%\thispagestyle{empty}

%Kontaktmöglichkeiten des Autors und der Prüfer
%\begin{center}

%Autor der Bachelorarbeit und die Prüfer
\begin{tabular}{rl}
        eingereicht von:  &  Birgit Schlotter\\
       			    	  &  Humboldtstraße 15\\
			      		  &  50676 Köln\\
			      		  &  Matrikel-Nr.: 5587077\\
						  &  Tel.: 01 77 - 6 41 66 05\\
						  &  E-mail: bschlott@smail.uni-koeln.de\\
						  &  1. Semester\\[0.8em]
\end{tabular}



%\begin{tabular}{rl}
%						
%	eingereicht von: &   Birgit Schlotter\\
%			&  Humboldtstraße 15\\
%			&  50676 Köln\\
%			&  Tel.: 01 77 - 6 41 66 05\\
%			&  E-mail: bschlott@smail.uni-koeln.de\\
%			&  1. Semester\\[0.8em]
							%&  \quad 1. Semester\\[2.0em]
							
%\end{tabular}
							
			
							
							
\end{center}

\end{titlepage}

\clearpage

\singlespacing %Inhaltsverzeichnis nur mit 1-Zeiligem Abstand
\tableofcontents  %Inhaltsverzeichnis
\thispagestyle{empty}
\clearpage

\onehalfspacing

%!TEX root = ../hausarbeit.tex
\section{Einleitung}

Wer kennt sie nicht, die Waldorfschule. Die Schule, an denen die Schüler ihren Namen tanzen oder die esoterische Baumschule auf die nur Hippies gehen. In Deutschland gibt es viele Vorurteile gegenüber dieser Schulform. Auch in meiner Stadt, in der ich aufgewachsen bin, gab es eine Waldorfschule. Ich kannte zwar Niemanden, der dort unterrichtet wurde, aber trotzdem haben wir über diese unbekannten Schüler gesprochen und vor allem gelacht. 

Schaut man sich diese Schulform aber genauer an, erkennt man schnell, dass die Schüler nicht nur ihren Namen tanzen und auch keine Hippies sind. Waldorfschulen sind irgendwie ganzheitlich, die Schüler besonders sozial, musikalisch und handwerklich geschickt. Viele Eltern suchen Alternativen zum staatlichen Schulsystems und melden ihre Kinder deshalb an dieser Schule an. Es wird viel Wert auf das fächerübergreifende Lernen gelegt sowie auf Motivation statt auf Leistungsdruck. 

Rudolf Steiner gründete 1919 die erste Waldorfschule. Über Sinn und Unsinn seiner Reformpädagogik kann man sich auch heute noch streiten. Kritiker sehen die Wissensvermittlung zu kurz kommen, Befürworter loben dagegen, dass die Schüler fürs Leben und nicht nur für die Zeugnisse lernen. Aufgrund von vielen Initiativkreisen zur Begründung von Waldorfschulen werden immer mehr dieser alternativen Schulen gegründet. Der Erfolg spricht also für sich und man muss einsehen, dass man diese Pädagogikform nicht mehr ignorieren kann und ausdrücklich thematisieren muss. 

Inzwischen gibt es bereits mehrere empirische Studien zur Waldorfpädagogik. Wie schneidet diese im Vergleich zu Regelschulen ab? Hält die Praxis, was Steiner einst erreichen wollte? Was wurde aus ehemaligen Waldorfschülern? Diese und weitere Fragen sollen in den unterschiedlichen Studien beantwortet werden. Dennoch sollte man diese auch kritisch hinterfragen. Nicht immer ist das sogenannte schwarz/weiss sehen möglich. 

In der folgenden Hausarbeit möchte ich mich deshalb mit der empirischen Studie zur Bildung und Lebensgestaltung ehemaliger Waldorfschüler von Heiner Bartz und Dirk Randoll auseinandersetzen und hinterfragen. Zu Beginn werde ich auf die Biographie Rudolf Steiners, den Begründer der Waldorfpädagogik, eingehen. Um zu verstehen, wie die Waldorfschulen aufgebaut sind, werde ich darauffolgend deren Entstehungsgeschichte und die heutige Umsetzung und Organisation erläutern. Zum Schluss werde ich die Wirkung der Waldorfpädagogik im Licht der Empirie darstellen.
%!TEX root = ../hausarbeit.tex
\section{Rudolf Steiner}
\subsection{Die frühen Jahre}
Rudolf Steiner wurde am 27.02.1861 in Kraljevek (ehemals Österreich, heute Kroatien) geboren. Er konnte dank eines Stipendiums Mathematik und Naturwissenschaften an der Wiener Technischen Hochschule und nebenbei noch Literatur, Philosophie und Geschichte studieren \citep[vgl.][]{wikipedia}.

Steiner promovierte 1891 zum Doktor der Philosophie an der Universität Rostock mit einer Arbeit über die Grundlagen der Erkenntnistheorie. Diese Arbeit wurde später erweitert und erschien dann als Buch unter dem Titel \emph{Wahrheit und Wissenschaft. Vorspiel einer Philosophie der Freiheit}. Sein Werk \emph{Philosophie der Freiheit}, was meist als sein Hauptwerk angesehen wird, entstand unter anderem in dieser Zeit \citep[vgl.][]{steiner}.

Steiner begann eine schriftstellerische Tätigkeit und gab verschiedene literarische Zeitschriften heraus. Von 1882 bis 1897 war Steiner z.B. Herausgeber der naturwissenschaftlichen Schriften Johann Wolfgang von Goethes. Daneben gab er aber auch Werke von Schopenhauer und Jean Paul heraus. Zuerst wurden seine herausgegebenen Bände von Goethe außerordentlich gelobt. Allerdings wurde dann recht früh bemängelt, dass er nicht Goethes Weltanschauung sondern seine eigene darstellte \citep[vgl.][]{steiner, wikipedia}.

Wie in \citet{hemleben63} beschrieben, erhielt Steiner im Jahr 1900 die Aufforderung, in der Theosophischen Bibliothek an zwei Abenden vor der Theosopischen Gesellschaft zu referieren. Dadurch hatte er ein Forum gefunden, vor dem er \enquote{die Fundamente seines zukünftigen Wirkens legen konnte} \site{hemleben63}{78}. In den beiden Wintern 1900/01 und 1901/02 folgten weitere Vorträge vor dem selbem Hörerkreis. Die Inhalte dieser Vorträge wurden in den Büchern \emph{Die Mystik im Aufgange des neuzeitlichen Geisteslebens und ihr Verhältnis zur modernen Weltanschauung} und \emph{Das Christentum als mystische Tatsache} veröffentlicht, welche als wesentliche Vorboten der Anthroposophie gelten.

\subsection{Die späten Jahre} % (fold)
\label{sub:biographie_steiner_in_seinen_sp_ten_jahren}

In den Jahren 1912/13 kam es wegen Steiners Stellung zum Christentum zum Bruch mit der indisch-angelsächsischen Theosophie \citep[vgl.][S. 80]{hemleben63}. Steiner hat \enquote{in Jesus Christus und dem 'Ereignis von Golgatha' das zentrale Geschehen der Erd- und Menschheitsgeschichte gesehen.} \site{hemleben63}{80} Dieser Ansicht waren die Theosophen nicht. Nach dem Rausschmiss Steiners von der theosophischen Gesellschaft 1913 fühlte er sich genötigt, die anthroposophische Gesellschaft zu gründen.

1919 bat Emil Molt, Direktor der Zigarettenfabrik Waldorf-Astoria, Rudolf Steiner um Unterstützung bei der Gründung einer Betriebsschule für die Kinder seiner Arbeiter. Dadurch entstand die erste Waldorfschule in Stuttgart. \citet[S. 124]{hemleben63} beschreibt, dass die Waldorfschule \enquote{zunächst als Betriebsschule einer Zigarettenfabrik - in Wirklichkeit als fruchtbarer Keim einer Weltschulbewegung} entstand. Steiner hielt in der Zeit von 1919 - 1924 etliche Vortragszyklen für Lehrer und Erzieher in Deutschland, England und Holland. Er vermittelte pädagogische Leitideen auf Grund seiner anthroposophischen Menschenkunde. 

1924 wurde von linker, okkulter Seite ein Giftanschlag auf Steiner verübt. Im September erlitt er einen physischen Zusammenbruch und sein Krankenlager begann. Währenddessen setzte er aber seine Arbeit an seiner Autobiographie \emph{Mein Lebensgang} fort, die er allerdings nicht beenden konnte. Am 30. März 1925 starb Rudolf Steiner in Dornach in der Schweiz \citep[vgl.][]{karl12}.










% The natbib package has two basic citation commands, \citet and \citep for textual and parenthetical
% citations, respectively. There also exist the starred versions \citet* and \citep* that print the full author list, and not just the abbreviated one. All of these may take one or two optional arguments to add some text before and after the citation.
       %Wie schon in \citet{geuenich09}
       %Wie schon in \citet[chap.~2]{geuenich09}
      %Wie schon in \citep{geuenich09}
      %Wie schon in \citep[chap.~2]{geuenich09}
      %Wie schon in \citep[see][]{geuenich09}
      %Wie schon in \citep[see][chap.~2]{geuenich09}
      %Wie schon in \citet*{geuenich09}
      %Wie schon in \citep*{geuenich09}



%Wie schon in \site{geuenich09}{213} beschrieben...\\
%Wie schon in \citep{steiner} beschrieben...

%Wie schon in \citep{wikipedia} beschrieben...





%!TEX root = ../hausarbeit.tex
\section{Die Waldorfschule}

\subsection{Die Entstehungsphase} % (fold)
\label{sub:die_entstehungsphase}

% subsection die_entstehungsphase (end)
Die Waldorfpädagogik ist nicht aus einem einheitlichen Guss entstanden, sondern in einem fortlaufenden Prozess. Sie wurde von Steiner auf Basis seiner bisherigen Gedanken zur Theosophie und Anthroposophie entwickelt. Ausschlaggebend war das Drängen von Emil Molt, der, wie in Kapitel \ref{sub:biographie_steiner_in_seinen_sp_ten_jahren} auf Seite \pageref{sub:biographie_steiner_in_seinen_sp_ten_jahren} beschrieben, eine Betriebsschule für die Kinder seiner Arbeiter gründen wollte. \citet[S. 125]{hemleben63} schrieb, dass Steiner \enquote{die Schule zu seiner Herzensangelegenheit} machte. Er nutzte die Anthroposophie zum Aufbau seiner Erziehungskunst.

Schon sehr bald erhielt das Konzept von Steiner die Zustimmung des Kultusministers aus Württemberg. Ihm gefiel besonders die Idee der Gesamtschule. Hier sollte auch den Schülern der Arbeiterklasse eine vollwertige Bildung geboten werden. \citep[Vgl.][S. 279]{frielingsdorf12} 

In den darauf folgenden Jahren wurden weltweit weitere Waldorfschulen gegründet. Wie in \citet{wikipedia2} beschrieben, gab es schon 1928 Waldorfschulen in Basel, Budapest, London, Lissabon und New York. In der Zeit des Nationalsozialismus lösten sich in Deutschland sechs Schulen selbst auf, die verbleibenden drei Waldorfschulen wurden bis 1941 geschlossen. Die Unterrichtsmethoden hätten  keine Gemeinsamkeit mit den nationalsozialistischen Erziehungsgrundsätzen, hieß es im Dekret von Reinhard Heidrich. Ab den 70er Jahren gab es dann Wieder- und Neugründungen in großem Umfang sowohl im In- als auch im Ausland. \citep[Vgl.][]{wikipedia2} 


\subsection{Die Entwicklungsphasen nach Steiner} % (fold)
\label{sub:Ziele, Besonderheiten, Rolle von Lehrern und Schülern}

Die Waldorfpädagogik erschließt sich vom Kindergarten bis zum Abitur. Steiner teilt die Entwicklungsphasen in drei Jahrsiebte ein und beschreibt diese wie folgt \citep[S. 139]{steiner10}:

\begin{quotation}
			\emph{\enquote{Bis zum Zahnwechsel will der Mensch nachahmen, bis zur Geschlechtsreife will er unter Autorität stehen; dann will er sein Urteil auf die Welt anwenden.}}
\end{quotation}

\citet{kiersch07} schreibt, dass für die Waldorferzieher das Kind im ersten Jahrsiebt (0 - 7 Jahre) kein unfertiger Mensch ist, dem Verhalten und Informationen eingeprägt werden muss. Das Kind ahmt aber das Verhalten des Erziehers sehr stark nach. Dies beeinflusst seine physischen Organe nachhaltig. Im zweiten Jahrsiebt (7 - 14 Jahre) kann man noch nicht auf Verstandsbeurteilung der Kinder bauen, sondern man muss herausfinden, wie einem das Kind glaubt, was man ihm als wahr, gut und schön vermittelt. Das Kind ist noch auf den Lehrer angewiesen, der ihm mit seiner Persönlichkeit die Welt erschließt. Im dritten Jahrsiebt (14 - 21 Jahre) entwickelt das Kind ein selbstständiges Urteil. Der Erzieher wird nun zum eigentlichen Lehrer und führt die Schüler unmittelbar an die Welt heran. 

\subsection{Organisation} % (fold)
\label{sub:organisation}

Die Schüler werden in der Regel von der ersten bis zur zwölften Klasse gemeinsam unterrichtet. Um den staatlichen Anforderungen gerecht zu werden, gibt es an den meisten deutschen Schulen eine zusätzliche Klasse, um die Schüler auf die Abiturprüfungen vorbereiten zu können. In der Waldorfschule gibt es kein Sitzenbleiben, so dass die Klassen nicht auseinandergerissen werden müssen. Soziales Verhalten wird in hohem Maße geübt. Es wird vermittelt, dass der begabtere Schüler dem etwas unbegabteren helfen solle. So lernt dieser auch direkt, seine sozialen Kräfte richtig einzusetzen. Viel Wert wird auch auf ein funktionierendes Vertrauensverhältnis von Lehrern und Eltern gelegt. So gehören Elternabende zu den regelmäßigen Einrichtungen der Waldorfschulen. \citep[Vgl.][]{kiersch07, hemleben63}

Während der ersten acht Schuljahre hat die Klasse einen Klassenlehrer, der den gesamten Hauptunterricht\footnote{Im Hauptunterricht werden i.d.R. die Kernfächer Deutsch, Mathematik, Geometrie, Geschichte und Sozialkunde, Biologie, Chemie, Physik, Astronomie und Kunstbetrachtung bearbeitet. \citep[Vgl.][S. 50]{kiersch07}} erteilt. Dadurch kann der Lehrer viel engere menschliche Bindungen mit den Schülern eingehen und es bildet sich beim Lehrer eine subtilere Kenntnis der Schüler, als dies durch das Fachlehrersystem möglich wäre. Die Klasse wächst zu einer engen Gemeinschaft zusammen. Erst ab der neunten Klasse wird der Unterricht von unterschiedlichen Fachlehrern erteilt. Jetzt beginnt in der Regel auch erst die Differenzierung nach Begabungen oder beruflichen Interessen. Dies wirkt sich in der Wahl der Fremdsprachen und den praktischen Fächern aus. Die Schüler bleiben in den Hauptfächern und in mehreren künstlerischen Fächern weiterhin zusammen. So wird dem Ideal von Steiner entsprochen, welches er 1919 entworfen hat, so dass zukünftige Arbeiter und Akademiker zusammen unterrichtet werden. Mehrere Schulen arbeiten daran, Praktika und berufsvorbereitende Grundkurse einzuführen, um denjenigen Schülern den Übergang ins Berufsleben zu erleichtern, die keine wissenschaftliche Ausbildung anstreben. Manche bieten auch eine handwerkliche Lehre an, die auf den übrigen Ausbildungsplan abgestimmt ist.  \citep[Vgl.][]{kiersch07, hemleben63}

\subsection{Unterrichtsstruktur und Beurteilungen} % (fold)
\label{sub:unterrichtsstruktur_und_Beurteilungen}
Die Fächer des Hauptunterrichts werden in der Regel während der gesamten Waldorfzeit in den ersten zwei Unterrichtsstunden ohne Pause gelehrt. Sie werden abwechselnd in Epochen von drei bis vier Wochen unterrichtet. Fremdsprachen oder auch andere Fächer, die permanentes Üben erfordern\footnote{Hierunter fallen z.B. handwerkliche und musische Fächer, Gartenbau, Eurythmie, Sport und Religion \citep[Vgl.][]{waldorfschule13}}, werden anschließend in Fachstunden nach einem festen Wochenstundenplan gelehrt. Dadurch soll dem natürlichen Tagesrhythmus des Menschen gefolgt und ein Ausgleich geschaffen werden. Der rhythmische Wechsel der Ansprüche und der Einsatzmöglichkeiten soll sich sehr positiv auf den Lernfortschritt auswirken und ein zerstückelter Lehrplan wird vermieden. Um die Schüler durch Hören in eine Sprache einzuführen, wird an Waldorfschulen bereits ab der ersten Klasse Englisch und Französisch gelehrt. \citep[Vgl.][]{kiersch07, hemleben63} 

An den Waldorfschulen werden keine Zensurzeugnisse ausgestellt. Wie \cite{geuenich09} schrieb, erteilte Steiner diesen eine klare Absage. Die Waldorfzeugnisse zeichnen sich deshalb durch schriftliche Beurteilungen ohne Zensuren aus. 
\todo{Müsste es nicht 'Wie ... \textbf{schreibt}, erteilte...' heissen, also im Präsenz, wenn man zitiert???}

Einmal jährlich zum Schuljahresende erhält jeder Schüler eine ausführliche Gesamtbeurteilung durch den Klassenlehrer. Dieses wird durch einige Fachzeugnisse ergänzt. Ab der neunten Klasse erhält jeder Schüler Einzelbeurteilungen der Fachlehrer. Es kann jedoch auf besonderen Wunsch hin ein Notenzeugnis für abgehende Schüler erstellt werden. In den Zeugnissen soll nicht nur der erreichte Leistungsstand festgestellt, sondern auch die für jeden Schüler unterschiedlichen Ursachen für bessere oder schwächere Leistungen dargestellt werden. In diesem Zusammenhang werden auch Hinweise für die zukünftige Arbeit gegeben. \citep[Vgl.][S. 51]{kiersch07}

\todo{Manchmal schreibst du 'Notenzeugnis', manchmal 'Zensurzeugnis'. So weit ich weiß, sollte man in wissenschaftlichen Arbeiten lieber durchgehend einen Begriff verwenden.}

 \citet[S. 102]{geuenich09} beschreibt diesen \enquote{Impuls in die richtige Richtung} als Ansporn für das Kind. Jedem Schüler wird mit dem Zeugnis ein Spruch oder Gedicht übergeben, welches als \enquote{Richtschnur} für das ganze kommende Schuljahr dienen soll. Der Text soll auswendig gelernt und gegebenenfalls vor der Klasse aufgesagt werden. Dadurch soll das Kind \enquote{ganzheitlich [...] in seinem Entwicklungsstand und seinen Charaktereigenheiten wahrgenommen und angesprochen werden}. Anbei ein Beispiel eines Zeugnis-Gedichtes für Drittklässler, verfasst von \cite{kullak13}:

 \begin{quotation}
			\emph{\enquote{Fest richte sich mein Blick – aufs Richtige.\\
							Still wäg und wag mein Herz – das Wichtige.\\
							Schön werd von meiner Hand – das Tüchtige.}}
\end{quotation}

Durch die zensurlosen Zeugnisse kann man Problemen wie z.B. Mangel an Objektivität, Auslösen von psychischem und körperlichem Stress bei den Schülern und konkurrenzorientierter Beurteilung entgegenwirken. 



 









%!TEX root = ../hausarbeit.tex
\section{Die Waldorfpädagogik im Licht der Empirie}
\subsection{Empirische Forschungen zur Waldorfpädagogik} % (fold)
\label{sub:empirische_forschung}

% subsection empirische_forschung_an_waldorfschulen (end)

Seit nunmehr gut 90 Jahren ist die Waldorfpädagogik mit stetigem Wachstum in Deutschland vertreten. Dennoch wurde ca. 80 Jahre lang keine empirische Studie zu ihren Auswirkungen durchgeführt. In den Sozialwissenschaften wurden im Zuge der empirischen Wende\footnote{Der empirische Wendepunkt in der Erziehungswissenschaft begann mit der internationalen Studie PISA der OECD 2000, bzw. bereits 1996 mit den Timss-Studien der IEA für Mathematik. Seither werden Schüler vermessen, in Kompetenzstufen eingeordnet und in Bildungsstandards normiert \citep[vgl.][]{arp11}.} eine große Zahl von Forschungsarbeiten durchgeführt und publiziert. Diese beschäftigten sich mit Fragen der Erziehung und Bildung, zu den Bedingungen der Prozesse schulischen Lernens etc. Im waldorfpädagogischen Zusammenhang findet man in dieser Hinsicht aber vergleichsweise wenig. Der Bund der freien Waldorfschulen hat sich damals z.B. gegen eine Teilnahme an der PISA Studie entschieden, da \enquote{eine utilitaristische sowie eine auf Leistung und Wettbewerb ausgerichtete Bildung eher fremd ist} \citep[vgl.][S.127ff]{paschen10}.  Die einzelnen Schulen konnten allerdings selber entscheiden, ob sie daran teilnehmen wollen.

An was kann es liegen, dass es so wenige empirische Studien zur Waldorfpädagogik gibt? \cite[S. 127f]{paschen10} nennt dafür folgende ausschlaggebende Gründe:

	\begin{itemize}
    	\item Um der eigenen Reputation keine Schäden hinzuzufügen, wollen sich die Forscher nicht mit den anthropologischen Grundlagen der Waldorfpädagogik auseinander setzen, da diese zuweilen als unwissenschaftlich gelten.

    	\item Steiners Anweisungen zufolge soll jeder Waldorflehrer bestrebt sein, regelmäßig über seine eigene Praxis nachzudenken. Deshalb wird vielerorts davon ausgegangen, dass bereits genügend Praxisforschung durchgeführt wird.

    	\item Bisher gibt es in der Waldorfschulbewegung nicht viele Personen, die genügend forschungsmethodisches Wissen in den empirischen Sozialwissenschaften haben. Dadurch hat sich in der Vergangenheit kein größeres Bewusstsein für adäquate Fragestellungen entwickelt. 

    	\item Bei Traditionalisten gelten die Äußerungen und Empfehlungen Rudolf Steiners zu Fragen der Erziehung und Bildung als nicht hinterfragbar. Auch nach Meinung vieler Anhänger bedarf die Waldorfpädagogik keiner kritisch-rationalen oder gar empirischen Auseinandersetzung, da es ja eine geisteswissenschaftlich begründete Pädagogik ist.
    \end{itemize}

Erst vor ca. zehn Jahren hat sich die Waldorfbewegung gegenüber dem empirischen Paradigma geöffnet. Der Anstoß kam allerdings von außen, weshalb dies erst mit großer Skepsis und Zurückhaltung geschah. Die Öffnung aus einer inneren Überzeugung heraus fehlte. Bestrebungen z.B. durch die Alanus Hochschule in Alfter\footnote{Die Alanus Hochschule ist eine staatlich anerkannte Privathochschule für Kunst und Gesellschaft. Sie bietet verschiedene Möglichkeiten der Qualifizierung von zukünftigen und tätigen Waldorflehrern an \citep[vgl.][]{alanus13}.} fördern nun den Austausch zwischen Waldorfpädagogik und Erziehungswissenschaft. In deren \enquote{Arbeitskreis Empirische Forschung Waldorfpädagogik} werden zweimal im Jahr geplante, laufende und abgeschlossenen Forschungsarbeiten zu waldorfpädagogischen Fragen von Experten der Waldorfschulbewegung und verschiedener Universitäten vorgestellt und diskutiert \citep[vgl.][S. 128]{paschen10}.

\subsection{Absolventen von Waldorfschulen - Eine empirische Studie zu Bildung und Lebensgestaltung ehemaliger Waldorfschüler} % (fold)
\label{Studie}

Interessant ist die Frage, wie wirksam die Pädagogik Steiners wirklich ist. Das Charakteristikum der Waldorfpädagogik ist die Inanspruchnahme nachhaltiger Wirkungen im Blick auf eine erfolgreiche Lebensgestaltung. \citet[][S. 13]{randoll07} beschreiben die gelingende Lebensgestaltung wie folgt: \enquote{Von der Freude am beruflichen Engagement, über Verantwortungsbewusstsein für Gesellschaft und Umwelt bis hin zu positiven Einflüssen auf Lebensführung und Gesundheit im Alter reichen die Wirkungserwartungen}.

Die Studie \enquote{Absolventen von Waldorfschulen - Eine empirische Studie zu Bildung und Lebensgestaltung ehemaliger Waldorfschüler} will herausfinden, ob sich Anhaltspunkte für die Verwirklichung der oben genannten Erwartungen finden lassen.
Ziel der Studie war es, gesicherte Resultate zu erhalten, wie der Besuch von Waldorfschulen sich auf die Schüler ausgewirkt hat und was sie heute über diese Schulform denken.
Die Studie wurde unter der Leitung von Professor Heiner Barz von der Heinrich Heine Universität Düsseldorf und von Professor Dirk Randoll von der Alanus Hochschule Alfter durchgeführt. 
Für die Untersuchung wurden ehemalige Waldorfschüler dreier Altersgruppen eingehend befragt. 
Es wurden Berufskarrieren, Lebensorientierung, Religion und Gesundheit analysiert. 
Die Schüler waren zum Zeitpunkt der Erhebung (2005/06) 62-66, 50-59 und 30-37 Jahre alt \citep[vgl.][]{randoll07, paschen10}.

Die Erfassung von Veränderungen innerhalb der Waldorfschüler sowie unterschiedlichste Vergleiche zwischen der Population ehemaliger Waldorfschüler und der Grundgesamtheit ihrer gleichaltrigen Mitmenschen konnte durch die ausführliche standardisierte Befragungen dieser drei verschiedenen Absolventengenerationen ermöglicht werden. Durch den Rückblick der ehemaligen kristallisieren sich Stärken und Schwächen der von ihnen erlebten Waldorfpädagogik heraus. Zum Beispiel das Bevorzugen und Vernachlässigen von Interessensdomänen oder die Gefahr der Abschottung nach außen hin, also alles, was nicht die Waldorfgemeinschaft betrifft \citep[vgl.][S. 12]{randoll07}.

\subsection{Hauptergebnisse zur Studie von Barz und Randoll} % (fold)
\label{sub:hauptergebnisse}

% subsection hauptergebnisse_zur_studie_zu_bildung_und_lebensgestaltung_ehemaliger_waldorfsch_ler (end)

Die Ergebnisse beziehen sich auf relativ weit zurückliegende Schulerfahrungen. 
Der jüngste Jahrgang der Befragung verließ spätestens 1994 die Schule. 
Da sich die Waldorfpädagogik aber auf das von Steiner zurückgehende Erziehungs- und Unterrichtskonzept gründet, spielt dieser Einwand kaum eine Rolle. 
Denn demnach besitzt die Waldorfpädagogik ein zeitlos gültiges und von kurzfristigen pädagogischen Moden unabhängiges Bezugssystem \citep[vgl.][S. 16]{randoll07}.

\subsubsection{Die Berufswahl}
\label{subsub:Berufswahl}

Für die Studie untersuchten Anne Bonhoeffer und Michael Brater die Aspekte des gelernten Berufs sowohl der Befragten als auch deren Eltern. In \citet[][S. 16f]{randoll07} werden folgende Ergebnisse der Studie genannt: Der am häufigsten angegebene Beruf sowohl bei den ehemaligen Schülern als auch bei den Eltern ist Lehrer. Interessant dabei ist, dass die Personen zumeist Lehrer an staatlichen Schulen geworden sind. 
15,5\% der Mütter sind Lehrerinnen, davon aber nur 1,5\% an Waldorfschulen. Bei den Vätern gaben 14,2\% den Beruf des Lehrers an und nur 1,4\% davon an Waldorfschulen. Fast ein fünftel der jüngeren Waldorfschüler stammt demnach aus einem Lehrerhaushalt. Der hohe Prozentsatz der Lehrer an staatlichen Schulen zeigt, dass die Eltern dieser Schulform nicht vertrauen und ihre Kinder deshalb auf alternative Schulen, wie z.B. die Waldorfschule schicken.

Von den ehemaligen Schülern sind dann wiederum 14,6\% Lehrer geworden. 
In der Stichprobe ist der Anteil an Lehrern fünf mal so hoch wie in der restlichen Bevölkerung. 
Die Differenz ist bei Ärzten und Apothekern (7,7\%) sowie geistes- und naturwissenschaftlichen Berufen (9,5\%) noch höher. 
Während man erkennen kann, dass der Beruf des Lehrers etwas rückläufig ist, hat sich die Quote weiterer gesundheitlicher Berufe wie Masseur, Krankengymnast oder auch Krankenschwester deutlich erhöht. 
Die Quote von ehemaligen Waldorfschülern, die eine Akademikerlaufbahn einschlagen ist mit 46,8\% beträchtlich. 
68,7\% davon haben die (Fach-)Hochschulreife erworben. Bei den Vätern ehemaliger Waldorfschüler hatten 40\% einen Akademikertitel. 
Im Vergleich zum Bundesdurchschnitt von 12\% (Stand 2004) ist auch dieser Anteil enorm hoch. 
Sehr selten schlugen die Schüler eine Berufslaufbahn in der Gruppe der Warenkaufleute und Bürofachkräfte ein. 
Dies deutet auf eine wirtschaftsferne berufliche Orientierung hin. 
Aufgrund der in der Studie von Barz und Randall präsentierten Daten kann man die Waldorfschule als \enquote{Schule des Bildungsbürgertums} bezeichnen \citep[][S. 17]{randoll07}.

Trotz des Strickunterrichts für Jungen und Holz- und Metallarbeiten für Mädchen sind die Geschlechterrollen in den Berufen eher klassischer Natur. 
Darauf deutet zumindest die Berufswahl der ehemaligen Schüler hin. 
Bei Lehrern und Künstlern herrscht eine weibliche Dominanz vor, während Ingenieure und z.B. auch Tischler eher von den Männern dominiert werden. 
Für die Berufszufriedenheit spielen der Studie zufolge die äußeren Anreize wie Prestige, Freizeit oder Einkommen eine untergeordnete Rolle. 
Die ehemaligen gaben an, dass die Zufriedenheit am meisten damit zusammenhinge, seine eigenen Neigungen und Interessen verwirklichen zu können, sowie die Identifikation mit der Arbeit. 

\subsubsection{Die Lebensorientierung}
\label{subsub:lebensorientierung}

Im Weiteren analysierte Thomas Gensicke die zentralen Lebensorientierungen der ehemaligen Waldorfschüler. 
In \citet[][S. 17ff]{randoll07} werden dazu folgende Aspekte beschrieben: Lebensaspekte, die als wichtig angesehen werden, zeigen eine Grundhaltung, die sehr musisch und kulturell ambitioniert ist. Der Alltag gestattet es aber oft nicht, sie in der gewünschten Intention ausführen zu können. Ähnlich geht es ihnen dabei bei ehrenamtlichen Engagements und meditativen und kontemplativen Bedürfnissen. 
In der Gesamtbevölkerung zum Vergleich werden zwischenmenschlich-emotionale Aspekte als eher unauffällig betont. 
Schnelle Autos fahren, Sportveranstaltungen besuchen oder Fernsehen werden auf der Skala der Wichtigkeit als eher mäßig  beschrieben. 
Gensicke teilt die ehemaligen Waldorfschüler in drei verschiedene Typen ein: 
	\begin{compactitem}
		\item Die Kulturorientierten (31\%)
		\item Die Beziehungsorientierten (33\%)
		\item Die Hedonisten (36\%)
	\end{compactitem}

Die Kulturorientierten bevorzugen anspruchsvolle kulturelle und bildungsbezogene Aktivitäten\footnote{z.B. Museum, Oper, Theater oder Lesen \citep[vgl.][S. 17]{randoll07}.}. 
Diese Gruppe interessiert sich am ehesten für anthroposophische Themen. 
22\% davon gaben sogar an, das sie praktizierende oder engagierte Anthroposophen seien. 
Bei den Beziehungsorientierten steht das Mitmenschliche und Emotionale im Vordergrund. 
Sie wollen u.a. für andere Menschen da sein und es ist eine häusliche Do-it-yourself Orientierung festzustellen. 
Bei den Hedonisten kann man ein auf Körperlichkeit, Sport und Sexualität prägendes Einstellungsmuster erkennen. 
In dieser Gruppe kann man auch feststellen, dass sie dem Lebensgenuss stärker zugewandt sind. 
Nur 1\% der Hedonisten bekennen sich zur Anthroposophie. 
Das Alter innerhalb der drei Gruppen ist auch interessant. 
Man findet den Kulturorientierten Typus eher in den älteren Jahrgängen, während man den Hedonisten am ehesten in den jüngeren Jahrgängen findet. 
Auch die Geschlechterverteilung ist klassisch: die Hedonisten sind eher männlich besetzt (64\% der Befragten), während die beiden anderen Gruppen einen leicht weiblichen Überschuss haben. 
Wichtig zu erwähnen ist auch, dass 37\% der Kulturorientierten und 62\% der Hedonisten ihre Kinder \emph{nicht} auf eine Waldorfschule geben würden, während die Angaben zum Wohlbefinden auf der Waldorfschule innerhalb der Gruppen nur sehr leicht schwankt (von 87\% bis 92\%). 
Die Hedonisten kritisieren auch gängige Punkte wie die Vernachlässigung der Naturwissenschaften viel schärfer. 

\subsubsection{Die religiöse Orientierung}

Michael N. Ebertz hat sich für die Studie der religiösen Orientierung der ehemaligen Schüler gewidmet. 
In \citet[][S. 18f]{randoll07} werden seine Ergebnisse wie folgt zusammengefasst: In den erhobenen Stichproben wurden 31,3\% Protestanten, 9,4\% Mitglieder der Christengemeinschaft und genau so viele Katholiken gezählt. 
\citet[][S. 18]{randoll07} erklären den höheren Anteil an Protestanten damit, dass die Anthroposophie \enquote{eine stärkere innere Affinität zum Arbeitsethos des Protestantismus hat}. 
Interessant dabei ist wiederum, dass in der jüngsten Altersgruppe der Anteil der Katholiken schon bei 14,6\% liegt, was wiederum zeigt, dass die Hedonisten, wie in Kapitel \ref{subsub:lebensorientierung} beschrieben, sich kaum mehr zur Anthroposophie bekennen. 

Anhand der vorliegenden Daten kann man der Waldorfschule also nicht vorwerfen, dass sie zur Anthroposophie erziehe. 
\citet[][S. 19]{randoll07} schreiben, dass die Mehrheit der Ehemaligen ihr \enquote{indifferent oder skeptisch} gegenüber stehen. 
Sie gaben auch an, dass die Waldorfschule kaum eine aktive Rolle bei der Vermittlung anthroposophischer Überzeugungen spiele, sie aber nichts desto trotz eine \enquote{hohe religiöse und weltanschauliche Offenheit} besitze.

\subsubsection{Schulerinnerung und -beurteilung}
\label{subsub:Erinnerung}

In \citet[][S. 19f]{randoll07} bekommt man Einblicke in die Beiträge von Dirk Randoll, Heiner Barz und Sylva Panyr. 
Sie beschäftigten sich mit den Schulerinnerung und Schulbeurteilung der ehemaligen Waldorfschüler. 
Sie berichten, dass die ehemaligen Schüler sich in den Bereichen Rechtschreibung, Fremdsprachen und der Vermittlung von Fachwissen im Nachteil sehen. 
Jedoch finden sie sehr positiv, dass sie gelernt haben, Dinge zu hinterfragen und Zusammenhänge zu erkennen. 
Die ehemaligen Schüler bestätigen, dass sie sich auf der Waldorfschule größtenteils äußerst wohl gefühlt haben. 
Die Identifikation mit der ehemaligen Schule ist ein zentraler Befund. 
Die meisten betonen, dass sie dort eine \enquote{gute Grundausstattung fürs Leben} bekommen haben, wie z.B. eine positive Lebenseinstellung, ein grundlegendes Vertrauen in die eigenen Fähigkeiten, Selbständigkeit und Anpassungsfähigkeit. 
Sie berichten, dass sie einen guten Sinn für das soziale Miteinander entwickelt haben, welcher nicht durch \enquote{leistungsbezogene Konkurrenzgefühle} beeinträchtigt wurde. 
Allerdings bestätigen knapp 60\% der Ehemaligen, dass die Waldorfpädagogik zu wenig leistungsorientiert sei. 
Des weiteren wird häufig angegeben, dass die Waldorfschule bezüglich der ineffizienten Wissensvermittlung sowie der Ausklammerung von Leistungsaspekten eine gewisse Weltfremdheit besäße. 

In dem Bericht wird außerdem hervorgehoben, dass die Erfahrungen mit den Waldorflehrern in Bezug auf das Menschliche, das Unterstützende sowie die Sicherheit und Orientierung, die durch den Klassenlehrer gegeben ist, sehr positiv empfunden wurde. 
Allerdings wurden hinsichtlich der fachlichen Qualitäten des öfteren Zweifel geäußert. 
Negative Erfahrungen gab es auch mit \enquote{dogmatischen, mit strengen oder mit bigotten Waldorfpädagogen}. 
Der Epochenunterricht, die 12-jährige Klassengemeinschaft und der Verzicht auf Noten wurden zum größten Teil befürwortet. 















%!TEX root = ../hausarbeit.tex
\section{Fazit}

%Ich war Schüler an einer staatlichen Schule und kann somit nicht aus Erfahrung sprechen, wie es ist, an einer Waldorfschule unterrichtet zu werden. Meiner Meinung nach steht der Großteil der Menschen dem Neuen, vor allem aber dem Anderen sehr skeptisch gegenüber\footnote{Hierzu passt das Sprichwort von Oliver Hassencamp: Was der Bauer nicht kennt, frisst er nicht. [...] \citep[Vgl.][]{roschk13}}, was bei mir auch auf reformpädagogische Schulen zutrifft. Wie ich im Seminar und während des Schreibens dieser Hausarbeit aber  gemerkt habe, kann man sich von vielen Vorurteilen lösen, wenn man sich nur näher mit der Thematik beschäftigt.

Wie wirkt nun die Waldorfpädagogik im Licht der Empirie? Schaut man sich die Hauptergebnisse der beschriebenen Studie aus Kapitel \ref{sub:hauptergebnisse} oberflächlich an, ist die Waldorfpädagogik von vielen Vorteilen geprägt. Ein überdurchschnittlich hoher Anteil an Absolventen, welche eine Akademikerlaufbahn einschlagen, ein hohes Maß an persönlichem Wohlbefinden auf der Schule, die Schüler lernen fürs Leben, nicht für Noten und das soziale Mit- und Füreinander wird stark gefördert. %Wieso lernen die staatlichen Schulen also nicht von den Waldorfschulen?

Betrachtet man die Ergebnisse näher, profitiert die von Steiner gegründete Pädagogik aber auch von einer sicherlich ungewollten selektierten Schülerschaft. Wie in Kapitel \ref{subsub:Berufswahl} beschrieben, haben 40\% der Väter ehemaliger Waldorfschüler einen Akademikertitel. Während die staatlichen Schulen z.T. sehr gemischte Klassen bezüglich Herkunft und Sozialschicht unterrichten, haben es die Waldorfschulen mit einer homogenen Klientel zu tun, die eher aus der gehobeneren Bildungsschicht stammt und bei denen die Schulkosten kein größeres Problem darstellen\footnote{Während die Kosten des Schulbesuch an staatlichen Schulen der Steuerzahler übernimmt, müssen Eltern an Waldorfschulen im Durchschnitt ca. 150 € pro Monat bezahlen (Stand 2007). Das Schulgeld stieg seit dem Jahrtausendwechsel um rund 20\% \citep[vgl.][]{mannheim09}.}. Die Eltern können zum Einen ihre Kinder selbst gut fördern und zum Anderen auch hohe Bildungs- und Erziehungsansprüche an die Schule stellen, sowie die Arbeit der Lehrer vielfältig unterstützen. %Darauf können staatliche Schulen nicht in diesem Maße zurückgreifen. 

Anhand der Daten der beschriebenen Studie kann man sagen, dass die Waldorfschule hinsichtlich der Förderung sozialer und personaler Fähigkeiten sehr gut ist. Man muss sich aber auch die Frage stellen, ob neue didaktische Modelle zeitnah den Einzug in die Waldorfpädagogik einhalten oder ob und in welchem Ausmaß an den inzwischen 90 Jahre alten traditionellen Formen des Waldorfunterrichts festgehalten wird. Dies kann dazu führen, dass sie blind gegenüber der Realität wird und sich selber von Neuerungen ausschließt, wie z.B. beim Umgang mit Leistungsanforderungen und Konkurrenzsituationen, welche man in der späteren Arbeitswelt häufig antrifft. Nichts desto trotz kann man anhand der von Barz und Randoll zur Verfügung gestellten Daten feststellen, dass die Waldorfpädagogik sehr persönlichkeitsbildend wirkt. Die Schüler verlassen die Schule mit einem hohen Maß an Erfahrungen, Fertigkeiten, sozialen Fähigkeiten und vor allem mit viel Interesse. 

Würden den Waldorfschulen noch mehr öffentliche Gelder zur Verfügung gestellt werden, könnten mehr finanziell schwächere Menschen diese Schule besuchen und somit wiederum mehr Studien z.B. zum Verglich mit staatlichen Schulen durchgeführt werden. Meine bedeutendste Erkenntnis dieser Hausarbeit ist aber, dass sich die Waldorfschule der empirischen Wissenschaft gegenüber noch weiter öffnen muss, so dass sie auch von außen her wichtige Anstöße bekommen kann, wie sie sich weiter in die richtige Richtung entwickeln kann und somit auch stärker über ihre eigenen Methoden nachdenken muss. Außerdem können weitere Vorurteile und offene Fragen durch eine größere Zahl an empirisch ermittelten Daten aufgelöst werden. 






\pagebreak

%\listoffigures %Abbildungsverzeichnis
%\listoftables %Tabellenverzeichnis


% \appendix
\bibliography{literatur}

% The command \nocite{*} causes all items in the database to be included in the references, regardless of whether or not they are cited in the paper.
% \nocite{*}

\end{document}


